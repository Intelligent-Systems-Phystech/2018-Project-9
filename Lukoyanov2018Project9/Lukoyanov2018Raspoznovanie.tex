\documentclass[12pt, twoside]{article}
\usepackage{jmlda}
\newcommand{\hdir}{.}


\begin{document}

\title
    {Распознавание текста на основе скелетного представления толстых линий и сверточных сетей}
\author
    {А.\,С.~Лукоянов$^1$} 
\email
    {lukoyanov.as@phystech.edu}
\thanks
    {Работа выполнена при
     %частичной
     финансовой поддержке РФФИ, проекты \No\ \No 00-00-00000 и 00-00-00001.}
\organization
    {$^1$МФТИ}
\abstract{
Задача оптического распознования символов уже стала классической среди задач компьютерного зрения. Несмотря на то, что качество существующих моделей довольно высоко, каждый год выходит множество научных работ, посвященных именно класификации символов. Традиционно входом таких алгоритмов является растровое изображение, что, безусловно, имеет большую прикладную значимость чем классификация символов в векторном представлении. Тем не менее, существует более комплексный подход, при котором растровое изображение сначала переводится в векторное представление путем построения скелета сивола, то есть графовой структуры, а потом подается на вход обучаемой модели. 

Такой подход имеет несколько недостатков, один из них заключается в неприменимости традиционных сверточных нейронных сетей на графовых структурах.  В данной работе мы предлагаем способ свертывания графовых структур, позволяющий породить информативное описание скелета толстой линии. Так же в работе приводится сравнительный анализ архитектур, работающих непосредственно на растровом представлении символов и  архитектур, использующих графовое представление символов. Нам удалось добиться значимого повышения качества распознавания толстых линий за счет нового способа порождения их описаний.
	
\bigskip
\noindent
\textbf{Ключевые слова}: \emph {классивикация символов; распознование текста; графовые структыр; скелетное предствление; толстые линии; свертки.}
}

\titleEng
	[JMLDA paper template] % краткое название; не нужно, если полное название влезает в~колонтитул
    {Machine Learning and Data Analysis journal paper template}
\authorEng
	[F.\,S.~Author] % список авторов (не более трех) для колонтитула; не нужен, если основной список влезает в колонтитул
	{F.\,S.~Author, F.\,S.~Co-Author, and F.\,S.~Name} % основной список авторов, выводимый в оглавление
    [F.\,S.~Author$^1$, F.\,S.~Co-Author$^2$, and F.\,S.~Name$^{1, 2}$] % список авторов, выводимый в заголовок; не нужен, если он не отличается от основного
\thanksEng
    {The research was
     %partially
    	 supported by the Russian Foundation for Basic Research (grants 00-00-0000 and 00-00-00001).
    }
\organizationEng
    {$^1$Organization, address; $^2$Organization, address}
\abstractEng
    {This is the template of the paper submitted to the journal ``Machine Learning and Data Analysis''.
		
	\noindent
	The title should be concise and informative. Titles are often used in information-retrieval systems. Avoid abbreviations and formulae where possible.
	Please clearly indicate the last names and initials of each author and check that all names are accurately spelled. Present the authors' affiliation
	addresses where the actual work was done.
	Provide the full postal address of each affiliation, including the country name and, if available, the
	e-mail address of each author.
	Provide only institutional affiliation, department/division affiliation are not required.

	\noindent
	A concise and factual abstract is required.
	The purpose of the abstract is to provide a summary~of the paper enabling the reader to decide whether or not to read the full text.
    	The abstract should state briefly the purpose of the research, the principal results and major conclusions.
    	An abstract is often presented separately from the article, so it must be able to stand alone.
    	For this reason, References should be avoided, but if essential, then cite the author(s) and year(s).
    	Also, non-standard or uncommon abbreviations should be avoided, but if essential they must be defined at their first mention in the abstract itself.
    	The requirements on the size of the abstract is about 200--300 words.
    	It should be provided in the next structured manner:
	
	\noindent
	\textbf{Background}:	One paragraph about the problem, existent approaches and its limitations.
	
	\noindent
	\textbf{Methods}: One paragraph about proposed method and its novelty.
	
	\noindent
	\textbf{Results}: One paragraph about major properties of the proposed method and experiment results if applicable.
	
	\noindent
	\textbf{Concluding Remarks}: One paragraph about the place of the proposed method among existent approaches.
		
	\noindent
	Immediately after the abstract, provide 5-7 keywords, avoiding general and plural terms and multiple concepts (avoid, for example, ``and'', ``of'').
	Use keywords that are specific and that reflect what is essential about the paper.
	Use keywords from the abstract, introduction and conclusion.
	These keywords will be used for indexing purposes.
		
	\noindent
    	\textbf{Keywords}: \emph{keyword; keyword; more keywords, separated by ``;''}}

%данные поля заполняются редакцией журнала
\doi{10.21469/22233792}
\receivedRus{01.01.2017}
\receivedEng{January 01, 2017}

\maketitle
\linenumbers

\end{document}
