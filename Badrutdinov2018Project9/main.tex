\documentclass{llncs}
\usepackage[T2A]{fontenc}
\usepackage[utf8]{inputenc}
\usepackage[russian]{babel} 
\usepackage{graphicx}
\usepackage{natbib}

\title{Распознавание текста на основе скелетного представления толстых линий и сверточных сетей }

\author{Бадрутдинов К.И.}
\institute{Московский физико-технический институт (Государственный университет) \\ \email{badrutdinov.ki@phystech.edu}}

\begin{document}

\maketitle

\begin{abstract}
Данная работа посвящена распознаванию текста на изображении. Задачей является классификация букв латинского алфавита. Рассматриваются два подхода к её решению. Первый подход заключается в использовании свёрточных нейронных сетей, второй - в скелетном представлении толстых линий.
\end{abstract}

\textit{Ключевые слова: сверточные нейронные сети, распознавание символов, скелетное представление}

\end{document}
