\documentclass{llncs}
\usepackage[T2A]{fontenc}
\usepackage[utf8]{inputenc}
\usepackage[russian]{babel} 
\usepackage{graphicx} 
\usepackage{authblk}
\usepackage{natbib}
\usepackage[title]{appendix}

\title{Распознавание текста на основе скелетного представления толстых линий и сверточных сетей }

\author{Литовченко Л.А.}
\institute{Московский физико-технический институт (Государственный университет) \\ \email{litovchenko.la@phystech.edu}}

\begin{document}

\maketitle

\begin{abstract}
Данная работа посвящена вопросу применения скелетного представления растрового изображения в задаче распознования печатных символов. Сравнивается 2 подхода к решению задачи: классический с использованием свёрточных нейросетей небольшой глубины и с построением скелета изображениея, кодировской его в вектор и последующей обработкой.
\end{abstract}

\textit{Ключевые слова: сверточные нейронные сети, распознавание текста, скелетное представление, эмбеддинг графа}

\end{document}
