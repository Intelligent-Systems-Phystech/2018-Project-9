\documentclass{llncs}
\usepackage[T2A]{fontenc}
\usepackage[utf8]{inputenc}
\usepackage[russian]{babel}
\usepackage{amsfonts}

\righthyphenmin=3

\title{Распознавание текста на основе скелетного представления толстых линий и сверточных сетей }

\author{Якушевский Н.О.}
\institute{Московский физико-технический институт (Государственный университет) \\ 
\email{yakushevskii.no@phystech.edu}}


\begin{document}

\maketitle

\begin{abstract}
	В данной работе решается задача классификации симоволов на растровых изображениях. Рассматриваются два подхода к решению этой задачи. Во первом входными данными для свёрточной нейронной сети являются сами растровые изображения. Во втором предварительно строится скелет символа на растровом изображении - плоский граф, который затем переводится в вектор и подаётся на вход нейронной сети. Точности классификации этих двух подходов сравниваются.
\end{abstract}

\textit{Ключевые слова: распознавание текста, скелетное представление, сверточные нейронные сети, cкелетизация, graph embedding}

\section{Введение}

Распознование текста на изображении и, в частности, классификация символов на растровых изображениях - классическая задача для машинного обучения.

Как правило, считается, что наиболее удачным подходом к её решению является построение свёрточной нейронной сети, входными данными которой будут растровые изображения. Такой подход был применён в \cite{Lecun1998}, \cite{Simard2003}, \cite{Ciresan2011}.

Также существует альтернативный подход, в котором предварительно выделяется скелет символа на изображении. Процесс скелетизации фигуры представляет собой нахождение множества всех вписанных в эту фигуру окружностей. Центры вписанных окружностей - вершины графа, который и называется скелетом фигуры. Таким образом, получается представления растровых изображений в виде графов. Затем графы представляются в виде векторов. Необходимо сократить размерность векторного пространства до оптимального значения и подать векторы, соответствующие изображениям, на вход нейронной сети. Реализацию такого подхода можно найти в \cite{Kushnir2012}.

Существует множество алгоритмов скелетизации символов, некоторые из которых описаны в \cite{Baranov2003} и в \cite{Mestetskiy2009}, также как и последующего уменьшения количества признаков, например, те, что содержатся в \cite{Orlov2016}. В данной работе мы искали наиболее удачную комбинацию для построения точного классификатора символов на растровых изображениях, и сравнение его эффективности с традиционным решением.

Для того чтобы оценить точность классификатора, использовалась метрика accuracy на изображениях рукописных цифр из датасетов MNIST.

\section{Постановка задачи}

Введём множество бинарных изображений $\mathbb{I}$ символов и цифр из алфавита $\mathbb{A}$. Скажем, что существует такая функция $f: \mathbb{I} \rightarrow \mathbb{A}$, которая сопоставляет каждому изображени из $\mathbb{I}$ символ из $\mathbb{A}$.

Теперь обозначим множество плоских неориентированных графов $\mathbb{G}$ и введём функцию $a: \mathbb{I} \rightarrow \mathbb{G}$, которая сопоставляет каждому изображению из $\mathbb{I}$ граф из $\mathbb{G}$. $a$ определяет выбранный нами способ скелетизации символов на изображении.

Осталось ввести обозначение для множества наборов признаков $\mathbb{F}$ и определить функцию $b: \mathbb{G} \rightarrow \mathbb{F}$, которая сопоставляет каждому графу из $\mathbb{G}$ набор признаков из $\mathbb{G}$. $b$ задаёт признаковое описание скелета символа на изображении.

Тогда задачей нашей работы является нахождение функции $c$, такой чтобы при фиксированных $a$, $b$ функция $a \circ b \circ c: \mathbb{I} \rightarrow \mathbb{A}$ приближала функцию $f$, т.е. минимизировала функцию потерь $L$ - cross entropy loss.

\section{Эксперимент}

Был произвeдена пара экспериментов по классификации симовлов на бинарных изображениях из датасета MNIST.

В данной работе не рассматриваются алогритмы бинаризации изображений, т.е изображения считаются изначально бинаризоваными. Изображения символов из датасета MNIST изначально не бинаризованы, поэтому к изображениям предварительно был применен алогритм бинаризации.

Таким образом, получив бинарные изображения символов из датасета MNIST мы получили возможность построить над ними первый базовый эксперимент, в котором мы построили алгоритм классификации, основанный на свёрточных нейронных сетях. В результате была получена точность 0.9879.

Затем провели второй эксперимент, в котором применили метод, основанный на скелетизации изображений. На бинарных изображениях из датасета MNIST был выделен скелет символов при помощи алогритма X (не можем сказать, пока не скинули код). Таким образом, мы получили представление каждого изображения ввиде плоского неориентированного графа, каждая вершина которого имела 1, 2 или 3 соседей и радиус вписанной в символ окружности.

После этого нашей задачей было ввести признаковое описание скелетного представления. В качестве признаков были выбраны среднее, минимальное, максимальное и стандартное отклонение каждой из величин:
\begin{itemize}
\item координаты (x, y) каждой вершины;
\item радиус окружности каждой вершины;
\item координаты (x, y) вектора каждого ребра;
\item длина вектора ребра;
\item угол наклона ребра
\end{itemize}
А также количество вершин со степенями 1, 2, 3. И того получилось 31-мерное пространсто признаков. В качестве модели на признаках был выбран градиентный бустинг над решающими деревьями (XGBoost). Итого, мы получили точность 0.8807.

Для улучшения полученного нами результата была реализована сверточная нейронная сеть, в которой перед скрытым слоем к признакам из сверточного слоя добавлялись признаки скелетного представления. Таким образом, удалось достичь точности 0.9880.

\bibliographystyle{utf8gost705u}
\bibliography{biblio}

\end{document}