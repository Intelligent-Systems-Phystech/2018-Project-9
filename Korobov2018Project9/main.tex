\documentclass{llncs}
\usepackage[T2A]{fontenc}
\usepackage[utf8]{inputenc}
\usepackage[russian]{babel} 
\usepackage{graphicx} 
\usepackage{authblk}
\usepackage{natbib}
\usepackage[title]{appendix}

\title{Распознавание текста на основе скелетного представления толстых линий и сверточных сетей }

\author{Коробов Н.С.}
\institute{Московский физико-технический институт (Государственный университет) \\ \email{korobov.ns@phystech.edu}}

\begin{document}

\maketitle

\begin{abstract}
Задача распознавания текста на изображении - одна из традиционных задач машинного зрения. В более узкой постановке такая задача сводится к классификации изображений букв. Для решения этой задачи ранее было изучено множество нейросетевых (НС) подходов, в которых картинку чаще всего рассматривают, как растровое изображение, что важно с точки зрения прикладной значимости алгоритма. На вход таких НС приходят пиксели изображения для каждого канала. Однако существуют методы получения векторного предствления растрового изображения, например, скелетное представление. В таком случае, входом НС будет некое векторное представление графа т.н. "скелет".
В данной работе предлагается сравнение качества классификации букв латинского алфавита модели НС, обученной на растровых изображениях, с моделью НС, построенной над скелетным представлением букв. Кроме того, приведено сравнение точности классификации при использовании разных алгоритмов построения скелета и сворачивания графа.
\end{abstract}

\textit{Ключевые слова: сверточные нейронные сети, распознавание текста, скелетное представление, граф, сворачивание графа, растровое изображение}

\end{document}
