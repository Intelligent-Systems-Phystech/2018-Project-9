\documentclass[10pt,a4paper]{llncs}
\usepackage[utf8]{inputenc}
\usepackage{amsmath}
\usepackage{amsfonts}
\usepackage{amssymb}
% !TeX spellcheck = ru_RU
\usepackage[russian]{babel}
\usepackage{graphicx}
\usepackage{comment}
\usepackage{cite}
\usepackage{cmap}
\usepackage{setspace}
\usepackage{authblk}
\usepackage{tabularx,ragged2e,booktabs}
\newcolumntype{L}{>{\RaggedRight\arraybackslash}X} % ragged-right version of "X"



\begin{document}
\title{ Распознавание текста на основе скелетного представления толстых линий и сверточных сетей }
%Анализ времени доставки данных в в многошаговых сетях ALOHA с регулярнойструктурой
\author{
	П.Н.~Куцевол \\
	kutsevol.pn@phystech.edu\\
}
\institute{МФТИ}

\maketitle

\begin{abstract} 
	Задача распознавания изображений является одной из классических задач машинного обучения, а одна из наиболее распространенных подзадач - задача распознавания текста. В данной статье рассматривается два подхода в классификации букв латинского алфавита. Один из них - кластеризация растровых изображений с помощью нейронных сетей. Такая задача уже рассмотрена в ряде работ и получены нейронные сети, достаточно точно классифицирующие буквы алфавита. Другой подход заключается в представлении символов в виде графов и обучении сверточной нейронной сети, которая будет оперировать состояниями вершин и ребер графа. Один из вариантов представления изображения буквы графом - скелетное представление толстых линий, описание котрого можно также найти в литературе. В рамках данной работы построена нейронная сеть над растровыми изображениями, изучены алгоритмы перехода от растровых изображений к скелетным представлениям и сконструирована сверточная нейронная сеть над скелетными представлениями, распознающая символы. Один из путей дальнейшей работы - комбинация растрового и скелетного представления для увеличения точности классификации.
\end{abstract}





\end{document}             % End of document.