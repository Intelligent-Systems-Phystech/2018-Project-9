\documentclass{llncs}
\usepackage[T2A]{fontenc}
\usepackage[utf8]{inputenc}
\usepackage[russian]{babel} 
\usepackage{graphicx}
\usepackage{natbib}

\title{Распознавание текста на основе скелетного представления толстых линий и сверточных сетей }

\author{Тушин К.А.}
\institute{Московский физико-технический институт (Государственный университет) \\ \email{tushin.ka@phystech.edu}}

\begin{document}

\maketitle

\begin{abstract}
В работе рассматриваются подходы решения задачи распознования символов. Один из методов используюет сверточные сети для классификации изображений. Другой метод заключается в анализе графовых струтур с помощью скелетного представления, полученных по изображению. Так же приведены сравнения точности этих подходов и архитектур на датасетах MNIST и Chars74K.
\end{abstract}

\textit{Ключевые слова: сверточные нейронные сети, CNN, распознавание символов, скелетное представление, Graph embedding}

\section{Введение}
Распознования символов это классическая задача компьютерного зрения. Основным подходом в таких задачах это использование сверточных слоев в нейросетях~\cite{cnn_lecun}~\cite{cnn_appl}. В таких нейросетях на вход подается изображение а на выходе получаем вероятность принадлежности изображения к каждому классу. Существует другой подход, при котором растровое изображение переводится в векторное представление путем построения скелета символа, а потом подается на вход модели для предсказания к какому классу принадлежит изображение.

Скелетизация представляет собой процесс заполнения внутренностей символов кругами, центры которых - вершины графа, соединяются с ребрами графа. Было проведено много много исследований на эту тему. В работе~\cite{graphs_gen} обсуждается моделирование рукописного текста с помощью жирных линий. В работе~\cite{graphs_shape_comp} проводится сравнение формы бинарных растровых изображений на основе скелетизации. Однако, существует и альтернативные методы, как, например описанный в работе~\cite{graphs_alt_method}. В этой работе главной задачей было распрямление текстовых строк на основе непрерывного гранично-скелетного представления изображений. 

Для оценки качества работы алгоритма использовалась метрика accuracy
на датасете, MNIST и датасете Chars74k.

\begin{thebibliography}{1}

\bibitem{cnn_lecun}
	\BibAuthor{LeCun Y. et al.}
	 Convolutional networks for images, speech, and time series //The handbook of brain theory and neural networks. – 1995. – Т. 3361. – №. 10. – С. 1995.
	 
\bibitem{cnn_appl}
	\BibAuthor{Ciresan D. C. et al.}
 	Convolutional neural network committees for handwritten character classification //Document Analysis and Recognition (ICDAR), 2011 International Conference on. – IEEE, 2011. – С. 1135-1139.
 
\bibitem{graphs_gen}
	\BibAuthor{Клименко С. В., Местецкий Л. М., Семенов А. Б.}
	 Моделирование рукописного шрифта с помощью жирных линий //Труды. – 2006. – Т. 16.

\bibitem{graphs_shape_comp}
	\BibAuthor{Кушнир О. и др.}
	 Сравнение формы бинарных растровых изображений на основе скелетизации //Машинное обучение и анализ данных. – 2012. – Т. 1. – №. 3. – С. 255-263.
	 
\bibitem{graphs_alt_method}
	\BibAuthor{Масалович А., Местецкий Л.}
	 Распрямление текстовых строк на основе непрерывного гранично-скелетного представления изображений //Труды Международной конференции «Графикон», Новосибирск.–2006.–4 c.

\bibitem{mnist_original}
	\BibAuthor{LeCun Y., Cortes C., Burges C. J.}
	MNIST handwritten digit database //
	Available: 
	 \BibUrl{http://yann. lecun. com/exdb/mnist}. – 2010. – Т. 2.

\bibitem{mnist_sample1}
	\BibAuthor{Zhu D. et al.}
	 Negative Log Likelihood Ratio Loss for Deep Neural Network Classification //arXiv preprint arXiv:1804.10690. – 2018.
	 
\bibitem{mnist_sample2}
	\BibAuthor{Nair P., Doshi R., Keselj S.}
	Pushing the limits of capsule networks //Technical note. – 2018.

\bibitem{mnist_sample3}
	\BibAuthor{Hsieh P. C., Chen C. P.}
	 Multi-task Learning on MNIST Image Datasets. – 2018.

\end{thebibliography}

\end{document}


