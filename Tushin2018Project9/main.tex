\documentclass{llncs}
\usepackage[T2A]{fontenc}
\usepackage[utf8]{inputenc}
\usepackage[russian]{babel} 
\usepackage{graphicx}
\usepackage{natbib}

\title{Распознавание текста на основе скелетного представления толстых линий и сверточных сетей }

\author{Тушин К.А.}
\institute{Московский физико-технический институт (Государственный университет) \\ \email{tushin.ka@phystech.edu}}

\begin{document}

\maketitle

\begin{abstract}
В работе рассматривается два подхода решения задачи распознования символов. В первом методе используются сверточные сети для классификации изображений. Другой метод заключается в анализе графов с помощью скелетного представления, полученных по изображению. Каждый граф получает векторное представление и анализируется при помощи нейронной сети. Так же приведены сравнения точности этих подходов.
\end{abstract}

\textit{Ключевые слова: сверточные нейронные сети, CNN, распознавание символов, скелетное представление, Graph embedding}

\end{document}

