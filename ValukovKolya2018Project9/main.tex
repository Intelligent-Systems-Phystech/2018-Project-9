\documentclass{llncs} 
\usepackage[T2A]{fontenc} 
\usepackage[utf8]{inputenc} 
\usepackage[russian]{babel} 
\usepackage{graphicx} 
\usepackage{natbib} 

\title{Распознавание текста на основе скелетного представления толстых линий и сверточных сетей } 

\author{Валюков Н.Д.} 
\institute{Московский физико-технический институт (Государственный университет) \\ \email{valyukov.nd@phystech.edu}} 

\begin{document} 

\maketitle 

\begin{abstract} 
В данной статье рассматривается задача распознавания символов на изображении. Задача распознования текста является одной из классических задач машинного обучениия. Предлагается сравнить два подхода к решению этой задачи. Один из них - классический. Испульзуется кластеризация растровых изображений с помощью нейронных сетей. Это довольно распространенный подход, который не раз показывал свою эффективность на практике. Во втором методе предлагается графовое описание символов на основе скелетного представления толстых линий. 
\end{abstract} 

\textit{Ключевые слова: сверточные нейронные сети, распознавание символов, скелетное представление, графовое описание} 

\end{document}
