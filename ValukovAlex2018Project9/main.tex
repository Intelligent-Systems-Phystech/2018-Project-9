\documentclass{llncs}
\usepackage[T2A]{fontenc}
\usepackage[utf8]{inputenc}
\usepackage[russian]{babel} 
\usepackage{graphicx}
\usepackage{natbib}

\title{Распознавание текста на основе скелетного представления толстых линий и сверточных сетей }

\author{Валюков А.Д.}
\institute{Московский физико-технический институт (Государственный университет) \\ \email{valukov.alex@gmail.com}}

\begin{document}

\maketitle

\begin{abstract}
В работе рассматривается задача распознавания символов на изображении. Предлагается сравнить два подхода к решению этой задачи. 
Первый подход - классический. На вход свёрточной нейронной сети подаётся дискретное растровое изображение. 
Во втором предлагается классификация символов на основе скелетного представления толстых линий и граф эмбединга.
\end{abstract}

\textit{Ключевые слова: сверточные нейронные сети, распознавание символов, скелетное представление, граф эмбединг}

\end{document}
