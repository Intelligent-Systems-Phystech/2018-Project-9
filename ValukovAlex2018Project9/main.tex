\documentclass{llncs}
\usepackage[T2A]{fontenc}
\usepackage[utf8]{inputenc}
\usepackage[russian]{babel} 
\usepackage{amsfonts}

\title{Распознавание текста на основе скелетного представления толстых линий и сверточных сетей }

\author{Валюков А.Д.}
\institute{Московский физико-технический институт (Государственный университет) \\ \email{valukov.alex@gmail.com}}

\begin{document}

\maketitle

\begin{abstract}
В работе рассматривается задача классификации символов на изображении. Предлагается сравнить два подхода к решению этой задачи. 
Первый подход - классический. На вход свёрточной нейронной сети подаётся дискретное растровое изображение. 
Во втором предлагается классификация символов на основе скелетного представления толстых линий и граф эмбединга.
\end{abstract}

\textit{Ключевые слова: сверточные нейронные сети, распознавание символов, скелетное представление, cкелетизация, граф эмбединг}

\section{Введение}

Задача классификации символов в своей более общей постановке распознавания объектов на изображении актуальна и современна для машинного обучения. Например, её приложения можно найти в биометрии человеческого глаза для идентификации личности, или же в системах дополненной реальности.

В большом количестве научных статей, выпущенных на эту тему (\cite{Simard2003}, \cite{Ciresan2011}), утверждается, что наиболее удачным решением такого рода задач является построение свёрточной нейронной сети над изображениями. В этой работе мы постараемся опровергнуть это утверждение, предположив, что наилучшим подходом для описания формы объектов является скелетное описание (похожие утверждения есть в статье \cite{Kushni2012}).

Скелетное описание фигуры - это её представление в виде плоского графа (более подробно можно посмотреть в \cite{Mestetskiy2009}). Существует множество алгоритмов построения такого графа, но все они основаны на поиске множества окружностей, вписанных в фигуру. Центры этих окружносткей - вершины графа, который и называется скелетом фигуры. Получив скелетное представление фигуры, необходимо определить правильный набор признаков, описывающих граф, и представить каждый граф набором этих признаков. После чего построить нейронную сеть над признаковым описанием графов.

В данной работе были построен классификатор комбинирующий концепции свёрточных нейронных сетей и скелетного описания признаков. Получена точность классификации такого алогритма на датасете изображений рукописных цифр MNIST (так же, как и в работах \cite{Nair2018}, \cite{Hseih2018})

\section{Постановка}

Введём следующие обозначения:

\begin{itemize}
\item $\mathbb{A}$ - алфавит.
\item $\mathbb{I}$ - множество изображений c символами из $\mathbb{A}$.
\item $f: \mathbb{I} \rightarrow \mathbb{A}$ - функция однозначно сопостовляющая каждому изображению из $\mathbb{I}$ символ из $\mathbb{A}$.
\item $\mathbb{S}$ - множество скелетных представлений символов на изображении из $\mathbb{I}$.
\item $a_1: \mathbb{I} \rightarrow \mathbb{S}$ - функция однозначно сопостовляющая каждому изображени из $\mathbb{I}$ скелетное представление из $\mathbb{S}$. Задаёт выбранный алгоритм скелетизации изображений.
\item $\mathbb{F}$ - множество наборов признаков скелетных представлений из $\mathbb{S}$.
\item $a_2: \mathbb{S} \rightarrow \mathbb{F}$ - функция однозначно сопоставляющая скелетному представлению из $\mathbb{S}$ набор признаков из $\mathbb{S}$. Задаёт признаковое описание скелетного представления.
\end{itemize}

Тогда задачей будет построить такую функцию $a_1: \mathbb{F} \rightarrow \mathbb{A}$, чтобы минимизировать функцию потерь:
$$L(a_1 \circ a_2 \circ a_3, f)$$
Где L(x, y) - функция кросс энтропии двух функций $x$ и $y$ на выборке изображений $\mathbb{I}$.

\section{Эксперимент}

Был произвдена пара экспериментов по классификации симовлов на бинарных изображениях из датасета MNIST.

В данной работе не рассматриваются алогритмы бинаризации изображений, т.е изображения считаются изначально бинаризоваными. Изображения символов из датасета MNIST изначально не бинаризованы, поэтому к изображениям предварительно был применен алогритм бинаризации.

Таким образом, получив бинарные изображения символов из датасета MNIST мы получили возможность построить над ними первый базовый эксперимент, в котором мы построили алгоритм классификации, основанный на свёрточных нейронных сетях. В результате была получена точность 0.9879.

Затем провели второй эксперимент, в котором применили метод, основанный на скелетизации изображений. На бинарных изображениях из датасета MNIST был выделен скелет символов при помощи алогритма X (не можем сказать, пока не скинули код). Таким образом, мы получили представление каждого изображения ввиде плоского неориентированного графа, каждая вершина которого имела 1, 2 или 3 соседей и радиус вписанной в символ окружности.

После этого нашей задачей было ввести признаковое описание скелетного представления. В качестве признаков были выбраны среднее, минимальное, максимальное и стандартное отклонение каждой из величин:
\begin{itemize}
\item координаты (x, y) каждой вершины;
\item радиус окружности каждой вершины;
\item координаты (x, y) вектора каждого ребра;
\item длина вектора ребра;
\item угол наклона ребра
\end{itemize}
А также количество вершин со степенями 1, 2, 3. И того получилось 31-мерное пространсто признаков. В качестве модели на признаках был выбран градиентный бустинг над решающими деревьями (XGBoost). Итого, мы получили точность 0.8807.

Для улучшения полученного нами результата была реализована сверточная нейронная сеть, в которой перед скрытым слоем к признакам из сверточного слоя добавлялись признаки скелетного представления. Таким образом, удалось достичь точности 0.9880.

\bibliographystyle{utf8gost705u}
\bibliography{biblio}

\end{document}
