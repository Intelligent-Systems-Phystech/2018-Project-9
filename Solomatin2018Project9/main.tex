\documentclass{llncs}
\usepackage[T2A]{fontenc}
\usepackage[utf8]{inputenc}
\usepackage[russian]{babel} 
\usepackage{graphicx}
\usepackage{natbib}

\title{Распознавание текста на основе скелетного представления толстых линий и сверточных сетей }

\author{Соломатин А.А.}
\institute{Московский физико-технический институт (Государственный университет) \\ 
\email{solomatin.aa@phystech.edu}}

\begin{document}

\maketitle

\begin{abstract}
Задача распознавания символа на картинке является чрезвычайно распространенной задачей машинного обучения. В связи с этим существует несколько основных подходов к решению. В первом методе мы используем сверточные нейросети небольшой глубины для анализа изображения. В другом методе мы представляем символ в виде графа. Из скелета этого графа строится вектор признаков, который затем обрабатывается нейросетью. Также приводится сравнение точности классификации этих алгоритмов на датасете MNIST.
\end{abstract}

\textit{Ключевые слова: сверточные нейронные сети, CNN, распознавание символов, скелетное представление, Graph embedding}

\end{document}
